\chapter{Small World Networks} \label{chpmwn}

The dynamics associated with social interactions have been an object of fascination among
scientists for many years and are experienced by individuals on they every day lives. The
experience of meeting a complete strange with whom we have apparently little in common
and finding unexpected that we have a mutual acquaintance is one with most of us are
familiar with, \emph{it's a small-world}, we say. As a society we have become obsessed
with connectedness, people no longer have meetings, they network, we no longer have many
friends, we are well connected, but the truth of the matter is that we are struggling to
make sense of the complex networks that have recently infiltrated our lives, networks
that always existed for scientits but has never been so visible to ordinary individuals.

The social network we are embedded today have immense reach, a structure we can only
dimly perceive, its functioning bewilders us and its consequence affects everyone in the
planet, from big names to ordinary citizens, a native Indian in the Amazon rain forest or
an Eskimo. The connected age has left us confuse about the consequences of globalisation,
panicking by the spread of severe acute respiratory syndrome (SARS) and bird flu on the
fair East, complacent about the HIV/AIDS epidemic in the developing world, disoriented by
the Internet, worried about contagion in the financial market and terrified of Al Qaeda.




The experience of meeting a complete strange with whom we have apparently little in
common and finding unexpected that we have a mutual acquaintance is one with most of us
are familiar with, \emph{it's a small-world}, we say. As a society we have become
obsessed with connectedness and are struggling to make sense of the complex networks that
have recently infiltrated our lives, networks whose reach is immense, whose structure we
can only dimly perceive, whose functioning bewilders us and whose consequence affects
everyone in the planet, from big names to ordinary citizens, a native Indian in the
Amazon rain forest or an Eskimo.

The connected age has left us confuse about the consequences of globalisation, panicing
by the spread of severe acute respiratory syndrome (SARS) and bird flu on the fair East,
complacent about the HIV/AIDS epidemic in the developing world, disoriented by the
Internet, worried about contagion in the financial market and terrified of Al Qaeda.

Science itself reflects the networks zeitgeist, researches are only now beginning to
unravel the structure of complex networks, from the nervous system of simple organisms to
the spread of HIV around the world. The size of of these networks is often the problem:
26,000 genes in the human genome [x,x1], billions of people on Earth and billions of
pages on the Internet. Even if we were given the complete wiring diagram for any of there
networks (a list of all nodes and the connections between them) one would not know what
to compute, the mass of data would be overwhelming.

The theory of social networks has come a long way and is one the has seen extensive
development over the past four decades, yielding multiple measures both of individuals
significance such as centrality [freeman x] and network efficiency [Yamaguchi], which may
elucidate non-obvious such as ``key player" in an organisation or its optimal structure
for say, information diffusion. Frequently, however, this research assumes linear models
of social process such as Markov models of diffusion [x] and is generally applied to
networks that consist of a relatively small number of components [x]. While many of the
measures defined in the literature can in principle be applied to networks of arbitrary
size and structure, the computational costs of doing so may be prohibitive (e.g.
freeman's 1979 betweensess centrality) and the benefits are at any rate unclear if the
process of interest is inherently non-linear, as it the case for information (or disease)
contagion models involving threshold [Granovetter x] or refractory effects [Murray x].
Hence the problem of analysing efficiently the structure of extreme large networks (in
which components may easily number in the hundreds of thousands or more) and modelling
the effects of structure on non-linear dynamical process, remain relatively unexplored.

The traditional network theory is concerned with the relationships between individuals
and the pattern of interaction [x]. The precise nature of the individuals is down played,
or even suppressed, in hopes of uncovering deeper lows. A network theorist will look at
any system of interlinked components and see an abstract pattern of dots connected by
lines [x]. Its the pattern that matters, the architecture of relationships, not the
identities of the dots themselves. Viewed from these lofty heights, many networks,
seemingly unrelated, begin to look the same.

The study of complex networks is only the next logical step in a large journey, the quest
for a science of spontaneous order. The next step is to move to more general kinds of
connectivity, where neighbours are defined in an abstract sense that need not be
geographical. Just as spatial coupling between non-linear systems spawned a new form of
collective behaviour (self-sustaining spiral and scroll waves) that could not occur in
the simpler geometries [x], complex networks give rise to even richer forms of
self-organisation. in fact, complex networks are the natural setting for the one of the
most mysterious form of group behaviour facing science today in the connected age.

\section{The Small World Phenomenon}
The first evidence the the world might indeed be small was first presented by the
psychologist Stanley Milgram in 1967 [x]. Milgram was interested in a unresolved
hypothesis circulating in the sociological community of the day. The hypothesis was that
the world, viewed as an enormous network of social acquaintances, was in a certain sense
``small", that is, any one person in the world could be reached through a network of
friends in only a few steps ( compared with the size of the entire network).

\textbf{OPTIONAL}\emph{In 1967, Milgram \cite{Milgram1967} created what is today known as
the ``small world phenomenon''. He conducted a quantitative study of the structure of
social networks in the United States and concluded that on average a chain of six
acquaintances can connect everyone in the world, a result that has entered folklore as
the phrase ``Six degrees of separation''. Further studies have shown that any two
randomly chosen people can be connected by a small chain of intermediate acquaintances,
and this phenomena is referred to as the small-world effect \cite{Watts1998}}.

Real-world networks show a relatively high degree of clustering and have low average
distance between pairs of nodes within the network. A new model of the small-world
phenomena has been recently proposed by Watts and Strogatz \cite{Watts1998}, which allows
for the clustering properties of real social networks, in which the probability of two
people knowing one another is greatly increased if they have a common acquaintance. In
this model, the population lives in a ring lattice graph with \textit{n} vertices and
\textit{k} edges ``connections'' per vertex. We them rewire each edge with probability
\textit{p, \(0\leq p \leq 1\)}. This construction allows one to tune the graph between
regularity \textit{\(p=0\)} and disorder  \textit{\(p=1\)}, and thereby to probe the
intermediate region \textit{\(0<p<1\)} about which little is known. Figure
\ref{smallword} shows the random rewiring procedure for interpolating between a regular
ring lattice and a random network by tuning the probability  \textit{p}. In this example
for clarity, \textit{n = 20} ``vertices'' and \textit{k = 4} ''edges''.

\begin{figure}[h]
\begin{center}
\includegraphics[width=\textwidth]{smallworld}
\caption{Small-world networks random rewiring procedure} \label{smallword}
\end{center}
\end{figure}

Small world networks may play an important role in the study of the influence of network
structure upon the dynamics of many social processes such as disease spreading, formation
of public opinion, distribution of wealth, and transmission of culture traits.

\subsection{Empirical Studies}
The first empirical work was conducted by Milgram in 1967 [x] in the United States.
Milgram was principally renowned throughout the world for his remarkable and disturbing
work on the apparent submission of human ethical values to authority [x 1969], however he
also conducted a highly innovative and less controvert test of the small-world
hypothesis, which is thought to be the beginning of the today so called small-world
theory. In this experiment, Milgram sent a number of packets to agreeable ``sources" in
Nebraska and Kansas, with instructions to deliver these packets to one of two specific
``target" persons in Massachusetts. The targets were named and described in terms of
approximate location, profession and demography, but the sources were only allowed to
send the packets directly to someone they by first name. The objective was to get the
packets from source to target with as few of these ``first-name basis links" as possible.
For more detailed discussion and information on this study see [x].

The upshot of all this was that Milgram determined that an average of six intermediaries
was all that required to get such a letter across the intervening expanse of geography
and society, a result that has entered folklore as the phrase ``six degrees of
separation". A second study by Korte and Milgram [1970] used essentially the same method
to examine the length of acquaintances clans between white in Los Angeles and a mixed
white-black target population in New York and found similar statistics. Whether this
number is, in reality, too low or too high is a matter of debate, a detailed discursion
about Milgram's work and finding is given by xx [x]. In any event, Milgram seemed to have
demonstrated that whatever the precise number was, it was not very big, compared with the
overall magnitude of the population.

Although the study of social networks and their use as a tool for examining the structure
of societies already had a considerable history by the time Milgram did his initial
experiments [Mitchell 1969], but none of this work had looked at the question of path
length in the same light as had Milgram. It also seems that very little work of this
nature and scale has occurred since, even though results did (and still do) spier
considerable interest. In fact, it seems that more empirical effort has been devoted to
the lower-level question of the number of acquaintances that the typical person
possesses. Effort in this department have been made by Freeman and Thompson (1989),
Bernard et al (1989) and the inverse small-world [x], but it turns out to be a difficult
exercise. It seems unlikely that even if such a number and its variance could be
convincing determined for any given definition of acquaintance, that it would play nearly
so as important a role in the understanding of networks as a comparable advance in the
understanding of network structure.

Recent empirical work [x] has generated the claim to a wider range of nonsocial network.
However, much about the small-world hypothesis continues to be empirically
unsubstantiated. In particular individuals in real social networks have only limited
local information about global social network and therefore finding short paths
represents a non-trivial search effort [,x,x]. Moreover, and contrary to accepted wisdom,
experimental evidence for short global chain length is extremely limited [x,x]. Almost
all other empirical studies of large-scale networks [4 -9, 16-19] have focused either on
non-social networks or on crude proxies of social interaction such as scientific
collaboration [x], actors and films [x] and e-mail networks within single institution
[x].

The last attempt to empirical quantify the small-world effect on social networks has just
recently been published [x]. This study conducted a global social-search experiment in
which more than 60,000 e-mail users attempted to reach one of 18 target persons in 13
different counties by forwarding messages to acquaintances. Once again no much surprising
has come out relating the magic number of six acquaintances, the study concluded that
social searches can reach their targets in a median of five to seven steps, depending on
the separation of source and target teachability. The study also concluded that
successful social search is conducted primarily through intermediate to weak strength
ties, does not require highly connected ``hubs" to succeed, in contrast to unsuccessful
social search, which disproportionately relies on professional relationships and that
although global social networks are, in principle, searchable, actual success depends
sensitively on individual incentive, reenforcing the principle that network structure
alone is not everything.

\subsection{Theoretical Studies}
The theoretical research specific to the small-world phenomenon has start about the same
time as the empirical one by Milgram in the 1960s with the formulation and initial
mathematical investigation of the problem by Pool and Kochen [x, 1978]. Although their
results only appear in published form well after Milgram's experiment, the ideas have
been in circulation for some ten years beforehand. These authors made substantial
progress on the problem, estimating both the average number of acquaintances that people
posses and the probability of two randomly selected members of a society being connected
by a chain of acquaintances consisting of one or two intermediates. They developed these
approximation under a variety of assumptions about the level of social structure and
stratification present in the population and concluded (speculatively) that even quite
structured populations would have acquaintance chains whose characteristic path lengths
are not much longer than those of completely unstructured populations (where the
probability of A knowing C, given that A knows B, is independent of whether or not B
knows C). For a population about of the United States (in the 60s) and an estimated
average number of acquaintances per person of about a thousand, Pool and Kochen estimated
that any member of the population could be connected to any other with a chain of
associates consisting of at most two intermediaries (hence three degrees of separation).

The study of distances in social networks however, has begun over twenty-five years
before the publication of Pool and Kochen's work, with Anotol Rapoport and his colleagues
at the University of Chicago. In a series of papers in the 1950s and 1960s, published in
the ``Bulletin of Mathematical Biophysics", Rapoport and colleagues established the
theory of ``random-biased nets", which described the statistics of disease spreading
through populations with varying degrees of structure. Although the early researches did
make significant gains on the issue of the effective size of social networks, their
progress was hampered by a number of difficulties that arose from both the questions they
chose to ask and the methodologies they used to seek answer. The result of Pool and
Kochen are highly suggestive of the small-world property's holding true in real
societies, but their results are highly sensitive the assumptions about conditional
probability in different parts of the population. A more recent article by Kochen in
1989, reports little progress on the essential theoretical difficulty. It turns out that
this is a problem faced by all theoreticians who find themselves exploring systems that
operate in the intermediate regime between order and randomness [x]. The problem arises
in many fields, notable fluid dynamics and the dynamics of coupled, nonlinear oscillators
[x], but in terms of social networks, the only network whose statistical properties are
analytical tractable are those that are either completely ordered or completely
random[x].

Although these cases are at opposite extremes of the structural spectrum, they both share
the essential characteristic that their local structure mirrors (either exact or
statistically) their global structure and hence analysis based on strictly local
knowledge is sufficient to capture the statistics of the entire network. That is, in an
important sense, they ``look" the same everywhere [x].

A new model of the small-world phenomenon has been recently proposed by Watts and Strogtz
[x], which allows for the clustering properties of real networks, in which the
probability of two people knowing one another is greatly increased if they have a common
acquaintance. This new formulation has given new life to the small-world phenomenon
theory and created a new network model called small-world networks.

\subsection{Formal Definition of the Small-World}
The phases small-world phenomenon, small-world problem and six degrees of separation have
long been an object of popular fascination and anecdotal report. At this stage, it is
necessary to agree on precisely what is meant by the small-world phenomenon, what is
known about it and when such a thing should be surprising in the first place. In general,
there is no unique and precise definition to what one mean when he or she says that the
world is ``small", but in this thesis, ``small" means that almost every element of the
network is somehow ``close" to almost every other element, even those that are perceived
as likely to be far away. This conjecture between reality and perception is what makes
the small-world phenomenon surprising to us. But why should one perceive the world to be
anything other than small in first place? The answer to this is fourfold:
\begin{enumerate}
    \parskip=0pt
    \item The network is numerically large in the sense that the world contains
    \emph{N} $\gg$ 1 people. In real world, \emph{N} is on the order of billions.
    \item The network is sparse in the sense that each person is connected to
    an average of only \emph{k} other people, which is, at most on the order of
    thousands [x], hundreds of of thousands of times smaller than the world
    population.
    \item The network is decentralised in that there is no dominant central
    vertex to which most other vertices are directly connected. This implies a
    strong condition than sparseness: not only must the average degree \emph{k}
    be much less than \emph{N}, but the maximal degree \emph{k$_max$} over all
    vertices must also be much less than \emph{N}.
    \item The network is highly clustered in that most friendship circles are
    strong overlapping. That is, one expects that many of his or her friends
    are friends also of each other.
\end{enumerate}

All four criteria are necessary for the small-world phenomenon to exist and be
remarkable. If the world did not contain many people, them it would not be surprising if
they were all closely associated, e.g. as in a small town. If most people knew most other
people then, once again, it would no be surprising to find that two strangers had a
acquaintance in common. If the network were not clustered, that is, if each person close
their friends independently of any of their friends' choice, then it follows from random
graphs theory [Bollob�s 1985] that most people would be only a few degrees of separation
apart even for very large \emph{N}. In fact, a random graph is a close approximation to
the smallest possible graph for any given \emph{N} and \emph{k} (where \emph{k$_max$}
$\ll$ \emph{N} and the variance in \emph{k} is not too large)[x].

These criteria are also satisfied by the real world. Given that the population of the
planet is seven billions and that even the most generous estimates of how many
acquaintances an average person can have is only a few thousand [Kochen 1989], then the
first two criteria are likely to be satisfied. The last two conditions are harder to be
sure of and certainly harder to measure, but they also seem quite reasonable in the light
of everyday experience. Some people are clearly more significant players than others, but
even the most gregarious individuals are constrained by time and energy to know a tiny
fraction of the entire population. What significance these individuals have must be due
to other more subtle and interesting reason. Finally, while it might be difficult to
determine in practice how many of a given person's friends are also friends with each
other and even more difficult to measure this for a large population, common sense tells
us that whatever this fraction is, it is much larger than that which one would expect for
a randomly connected network. To be more explicit about this, if the world were randomly
connected, then one's acquaintances would be just likely to come from a different
country, occupation and socioeconomic class as one's own. Clearly this is not the case
among human in real life.

\subsubsection{Small-World Graphs}
Although regular networks and random graphs are both useful idealisation, many real world
networks lie somewhere between the extremes of order and randomness. Real-world networks
show a relatively high degree of clustering and have low average distance between pairs
of actors within the network. A new model of the small-world phenomenon has been recently
proposed by Watts and Strogatz [x], which allows to clustering properties of real social
networks, in which the probability of two people knowing one another is greatly increased
if they have a common acquaintance. In this model, the population lives in a ring lattice
graph with \emph{N} vertices and \emph{k} edges connections per vertex. They them rewire
each edge with probability \emph{p} (0 $\leq$ p $\leq$ 1). This construction allows one
to ``tune" the graph between regularity (\emph{p}=0) and disorder (\emph{p}=1) and
thereby to probe the intermediate region (0
 $<$ \emph{p} $<$ 1) about which little is known.

\subsection{Small-World in the Real World}

\section{Models of Small-World Networks}

\subsection{Quantifying Small-World Networks}
The small-world network model by Watts and Strogtz [x] has borrowed some definitions from
graph theory in order to make the notation precise. For simplicity, the networks are
represented as connected sparse graphs, consisting solely of undifferentiated vertices
and unweighed, undirected edges.

The first characteristic of interest for a given graph is the characteristic path length
\emph{L}, defined as the average number of edges that must be transverses in the shortest
path between any pairs of vertices in the graph. In terms of Milgram's experiment,
\emph{L} would be the chain length averaged over all possible sources in the network and
all possible targets. \emph{L} them is a measure of the global structure of the graph as
one requires information about the entire graph in order to define the shortest path
between any two vertices. By contrast, the clustering coefficient \emph{C} is a measure
of the local graph structure. Specially, if a vertex \emph{v} has \emph{k} immediate
neighbours, them this neighbourhood defines a subgraph in which at most
\emph{k$_v$}(\emph{k$_v$} - 1)/2 edges can exist (if the neighbourhood is fully
connected). \emph{C$_v$} is them the fraction of this maximum that is realised in
\emph{k}'s actual neighbourhood and \emph{C} is this fraction averaged over all vertices
in the graph. Equivalently, \emph{C} can be regarded as the probability that any two
vertices (\emph{u,v}) will be connected, given that each is also connected to a ``mutual
friend" \emph{w}.

Watts and Strogatz conjectured that these two properties (short paths and high
clustering) would hold also for many natural and technological networks. Furthermore,
they conjectured that dynamical system coupled in this way would display enhanced signal
propagation speed, synchronisability and computational power, as compared with regular
lattices of the same size. The intuition is that the short path could provide high-speed
communication channels between distant parts of the system, thereby facilitating any
dynamical process (like synchronisation or computation) that requires global coordination
and information flow.

Research on small-world networks has proceeded along several fronts since the publication
of Watts and Strogatz study. Perhaps the strongest response has come from statistical
physics, who sense immediately [10] that the Watts and Strogatz model would yield to
their techniques. The path length \emph{L} in the original model [x] was soon found to be
poorly formulated, since it is possible that some part of the graph might become detached
through the rewiring process and the distance of the particular part of the graph will
therefore become infinite [x]. A fix to this problem was suggested by Newman and Watts
[x], were in its improved form, the model starts with a ring of \emph{N} vertices, each
connected to its nearest \emph{k} and next-nearest neighbours out to some range \emph{k}.
Shortcut links are them added (rather than rewired) between randomly selected pairs of
vertices with probability \emph{p} per edge on the underlying lattice, thus there are
typically \emph{Nkp} shortcuts in the graph. Most of the recent work on models of
small-world networks have been performed using this variant, although both version are
referred as small-world network models or small-world graphs.

\begin{center}Figure x (complex network, figure 4) goes here \end{center}

When dealing with real social networks, there are four main limitation in both models:
\begin{enumerate}
    \parskip=0pt
    \item The fully connected property of the network, which does not match
    social networks, where people are likely to be unattached for some time;
    \item The assumption that vertices ``actors" are equally considered for
    rewiring or shortcut, which is not likely in the real world as the level of
    interaction that a individual possess will not be the same for other
    individuals;
    \item Neither model characterise the ``ties" properties of edges, which is
    essential to capture the social behaviour of individuals within the
    network;
    \item And finally, neither model provides a measurement of information
    propagation through the network connections, essential for the statistical
    analysis of the network efficiency.
\end{enumerate}

\subsubsection{Harmony and Efficiency of Small-World Networks}

\subsection{Classes of Small-World Networks}

\subsection{Epidemics in a Small-World}

The spread of STDs and, in particular, HIV results from a complex network of social
interactions and other factors related to culture, sexual behaviour, demography,
geography and disease characteristics, as well as the availability, accessibility and
delivery of public healthcare.

Mathematical models characterising the dynamics of STD transmission indicate that
heterogeneity in sexual activity allow them to persist even when the typical behaviour of
the population would not support endemicity. This suggests that more attention should be
focused on the properties of the distribution of sexual activity within a population
\cite{Watts1998}. The nodal properties such as gender, ethnicity or relationship status
are of fundamental importance for the formation of networks \cite{Newman2000}. In
addition, to deal with STDs models, other network properties like concurrency, which
allows simultaneous partnerships to be formed, play a fundamental role in epidemics on
social networks \cite{Newman1999}.

In order to accommodate differential selectivity, ``nodal properties'', sexual contacts
behaviour ``edge properties'' and network properties such as performance and concurrency,
fundamental for the modelling of human sexual contact network and the spread of STDs,
I've modified the original Watts and Strogatz \cite{Watts1998} topological model as
follows:
\parskip=0pt
\begin{itemize}
    \item Relaxed the connectedness restriction;
    \item Relaxed the fix number of edges per vertex to allow concurrency;
    \item Added properties for each vertex based upon their population;
    \item Added conditions that two vertices must hold in order to be connected;
    \item Added properties for each edge in order to catch sexual behaviour changes.
\end{itemize}
\parskip=\baselineskip
