% !TeX encoding = ISO-8859-1
% !TeX spellcheck = pt_BR

\begin{thesisresumo}
O presente trabalho � uma investiga��o sobre a teoria e as aplica��es dos sistemas generativos de projeto como ferramenta de aux�lio na tomada de decis�es para a concep��o de formas na ind�stria da constru��o civil. Adotando uma metodologia experimental, um conjunto de algoritmos foi proposto como instrumentos de concep��o e desenho de um grupo de treli�as de suporte de uma cobertura tipo \textit{shed}, inspiradas na obra do arquiteto Jo�o Filgueiras Lima (Lel�). Al�m da gera��o de formas, o sistema conta com uma ferramenta de an�lise de estruturas, cuja modelagem foi baseada na Grafost�tica. Uma metodologia de trabalho e abordagem das quest�es relativas ao projeto arquitet�nico � proposta, e as contribui��es deste m�todo para a concep��o de elementos construtivos s�o avaliadas e discutidas.




\end{thesisresumo}
\hfill \break

Palavas Chave: Sistemas Generativos de Projeto, Algoritmos Generativos, Modelagem Param�trica, Grafost�tica, Arquitetura.
